\chapter{Fazit}

In der Arbeit wurden verschiedene Varianten des Rucksackproblems betrachtet,
wie beispielsweise das $0-1$ Rucksackproblem, das Bounded Rucksackproblem oder das Multiple Rucksackproblem. 
Jede Variante bringt spezifische Herausforderungen und Anwendungsbereiche mit sich.
\\
In der Arbeit wurden des Weiteren verschiedene Lösungsansätze
in Form von Algorithmen für das Rucksackproblem untersucht.
Darunter exakte Lösungsmethoden wie dynamische Programmierung, Branch-Bound Algorithmen 
sowie approximative Algorithmen wie beispielsweise Greedy-Algorithmen. 
Dabei wurde festgestellt, dass exakte Lösungsverfahren die optimale Lösung bieten können, 
aber bei großen Problemgrößen aufgrund ihrer hohen Laufzeit ineffizient werden können. 
Approximative Algorithmen hingegen bieten eine gute Annäherungslösung in akzeptabler Zeit, 
jedoch ohne Garantie für die optimale Lösung.
\\

Abschließend lässt sich festhalten, dass das Rucksackproblem eine faszinierendes und vielseitige Aufgabe der diskreten Mathematik ist.
Es gibt zum jetzigen Stand keine universelle Lösung die für alle Varianten und Problemgrößen optimal ist. 
Die Wahl des richtigen Lösungsansatzes hängt von verschiedenen Faktoren ab, wie der Problemgröße, den Laufzeit oder der Genauigkeit der Lösung. 
Zukünftige Forschungen könnten darauf abzielen, neue effiziente Algorithmen oder Heuristiken zu entwickeln,
um Lösungen für das Rucksackproblem weiter zu verbessern und deren Anwendbarkeit auf reale Problemstellungen auszudehnen.