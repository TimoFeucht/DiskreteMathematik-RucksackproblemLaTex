%!TEX root = ../dokumentation.tex


\newcommand*{\dpd}{\ensuremath{\mathit{dp}}}

\chapter{Algorithmen zur Lösung des Rucksackproblems}
\section{Brute-Force Algorithmus}
Um den Brute-Force-Algorithmus mathematisch zu beschreiben, 
müssen zuerst die Eingabeparameter des Problems definiert werden. Sei 
$n$ die Gesamtanzahl der verfügbaren Gegenstände. Jeder 
Gegenstand $j$ hat einen bestimmten Profit $p_j$ und ein Gewicht $w_j$. 
Der Rucksack hat eine maximale Kapazität $c$.
Der Algorithmus lässt sich nun in fünf Schritten beschreiben:

\texttt{Schritt 1}
Zunächst werden alle möglichen Teilmengen der Liste berechnet. 
Das bedeutet, dass jede Kombination von Elementen gebildet 
wird, was insgesamt $2^n$ mögliche Teilmengen ergibt.

\texttt{Schritt 2:}
Für jede Teilmenge wird die Summe der Gewichte berechnet und 
überprüft, ob die Summe das Gewichtslimit  
überschreitet. Wenn die Summe das Gewichtslimit nicht 
überschreitet, wird die Teilmenge als gültige Lösung 
betrachtet, andernfalls wird sie verworfen.

\texttt{Schritt 3:}
Schritt 2 wird für jede Teilmenge wiederholt, bis eine 
Teilmenge gefunden wird, deren Gewicht das Gewichtslimit 
nicht überschreitet.

\texttt{Schritt 4:}
Die Werte der gültigen Teilmengen, die die Bedingung in 
Schritt 3 erfüllen, werden aufsummiert.

\texttt{Schritt 5:}
Schließlich wird der höchste Wert aus den gültigen 
Teilmengen als Antwort auf das Problem ausgewählt.\ \cite[vgl.][]{balogun2022explanatory}

Die Laufzeitkomplexität des Brute-Force-Algorithmus beträgt 
$O(n2^n)$, da der Algorithmus alle möglichen Kombinationen der 
Gegenstände durchläuft. Diese exponentielle Komplexität 
macht den Brute-Force-Algorithmus für große $n$ unpraktikabel.\ \cite[vgl.][]{hristakeva2005different}

\section{Branch-and-Bound Algorithmus}
Zuerst wird die Entscheidungsvariable $x_j$ für jeden 
Gegenstand $j$ definiert. Wenn der Gegenstand $j$ ausgewählt wird, ist $x_j = 1$, 
andernfalls ist $x_j = 0$. Darüber hinaus 
wird die Funktion $f(x) = \sum_{j=1}^{n}(x_j  p_j)$ zur Berechnung des 
Gesamtprofits einer Teilmenge von Gegenständen $x$ und die 
Funktion $g(x) = \sum_{j=1}^{n}(x_j w_j)$ zur Berechnung des Gesamtgewichts einer 
Teilmenge von Gegenständen $x$ verwendet.

Der Branch-and-Bound-Algorithmus beginnt mit einer 
Anfangslösung, die keinen Gegenstand enthält 
$(x = {0, 0, \ldots, 0})$. Anschließend wird ein Suchbaum 
aufgebaut, der alle möglichen Teilräume des Lösungsraums 
repräsentiert und die Gegenstände in $x$ absteigend nach dem Verhältnis 
von Profit zu gewicht sortiert. Jeder Knoten im Suchbaum entspricht einer 
Teilmenge von Gegenständen. Der Algorithmus durchläuft den 
Suchbaum rekursiv, indem er Teilräume basierend auf 
Schätzungen und Grenzwerten erzeugt und abschneidet.

Die Schätzungsfunktion $E(x)$ wird verwendet, um eine obere 
Schranke für den Wert einer Teilmenge von Gegenständen $j$ zu 
berechnen. Diese Schätzung basiert auf einer Relaxation des 
Rucksackproblems, bei der angenommen wird, dass beliebige 
Bruchteile von Gegenständen ausgewählt werden können. Eine 
mögliche Schätzungsfunktion ist:

\begin{equation}
    E(x) = f(x) + (c - g(x)) \cdot \left(\frac{p_{j+1}}{w_{j+1}}\right)
\end{equation}

\noindent wobei $j + 1$ der Index des nächsten Gegenstands ist, der nicht 
in $x$ enthalten ist. Diese Schätzung beruht auf der Annahme, 
dass der nächste Gegenstand mit dem höchsten Verhältnis von 
Wert zu Gewicht ausgewählt wird, um die verbleibende 
Kapazität optimal auszunutzen.

Basierend auf der Schätzungsfunktion wird ein Grenzwert 
(bound) für den aktuellen Teilraum berechnet. Der Grenzwert 
$B(x)$ wird definiert als der maximale Wert, der in einem 
Teilraum von Gegenständen $j$ erreicht werden kann, ohne das 
Gewichtslimit zu überschreiten. Er wird 
rekursiv berechnet, indem alle Teilräume des aktuellen 
Teilraums betrachtet werden. Wenn ein Teilraum das 
Gewichtslimit überschreitet, wird der Grenzwert auf den Wert 
des aktuellen Teilraums gesetzt. Andernfalls wird der 
Grenzwert als die maximale Summe der Werte aller Gegenstände 
im Teilraum plus der maximalen Summe der Werte, die aus dem 
Rest des Rucksacks ausgewählt werden können, berechnet.

Der Branch-and-Bound-Algorithmus verwendet eine Tiefensuche, 
um den Suchbaum zu durchlaufen. Bei jedem Schritt wird ein 
Knoten ausgewählt und seine Kinderknoten generiert. Die 
Kinderknoten entsprechen den Teilräumen, die durch 
Hinzufügen oder Nicht-Hinzufügen des nächsten Gegenstands 
entstehen. Die Kinderknoten werden nach ihrem Grenzwert 
sortiert, und nur die vielversprechendsten Knoten werden 
weiter erkundet, während unproduktive Knoten abgeschnitten 
werden.

Der Algorithmus terminiert, wenn alle Knoten des Suchbaums 
erkundet wurden. Die optimale Lösung ist der Knoten mit dem 
höchsten Gesamtwert. Der Branch-and-Bound-Algorithmus 
garantiert die Rückgabe einer optimalen Lösung.

Die Laufzeit des 
Branch-and-Bound-Algorithmus hängt stark von der Größe des 
Suchraums ab. Die beste mögliche Laufzeit ist 
$O(n)$, falls nur ein Ast des Baums durchsucht werden muss und 
die schlechteste $O(2^n)$.\ \cite[vgl.][]{Martello1987}

\section{Greedy-Algorithmus}
Der Greedy-Algorithmus zur Lösung des Rucksackproblems besteht 
aus fünf Schritten:

\texttt{Schritt 1:} Berechne das Verhältnis von Wert zu Gewicht für 
jeden Gegenstand.
Verhältnis \[r_j = \frac{p_j}{w_j}\]

\texttt{Schritt 2:} Sortiere die Gegenstände absteigend nach ihrem 
Verhältnis von Wert zu Gewicht.
\[r_1 \geq r_2 \geq \cdots \geq r_n\]

\texttt{Schritt 3:} Initialisiere den Rucksack $R$ und den Gesamtwert $P$.
\[   \text{$R$} = \{\}; \\
    \text{$P$} = 0 \]

\texttt{Schritt 4:} Durchlaufe die sortierte Liste der Gegenstände.\\
Für $j = 1$ bis $n$: 
Wenn das Hinzufügen des Gegenstands $j$ zum Rucksack den 
verbleibenden Platz im Rucksack nicht überschreitet:
Füge den Gegenstand $j$ zum Rucksack hinzu.
Aktualisiere den verbleibenden Platz im Rucksack.
Erhöhe den Gesamtwert um den Wert des Gegenstands $j$.

\texttt{Schritt 5:} \\Gib den Rucksack und den Gesamtwert als Ergebnis 
aus.

Der Greedy-Algorithmus basiert auf der Annahme, dass es 
immer optimal ist, den Gegenstand mit dem höchsten Verhältnis 
von Wert zu Gewicht zu wählen. Diese Annahme kann bei dem Rucksackproblem
keine optimale Lösung garantieren.

Um dies zu beweisen, betrachten wir ein einfaches Beispiel: 
Gegeben seien drei Gegenstände mit den Werten $p = \{60, 100, 120\}$ 
und den Gewichten $w =\{10, 20, 30\}$. Der Rucksack hat eine Kapazität 
von $c = 50$. Der Greedy-Algorithmus würde den Gegenstand $1$ und $2$ mit 
dem höchsten Verhältnis von Wert zu Gewicht auswählen (Gesamtwert des Rucksacks bei 160). Die Optimale 
Lösung lässt Gegenstand $1$ aus dem Rucksack und nimmt nur Gegenstand $2$ und $3$ (Gesamtwert bei 220).
Damit ist gezeigt, dass der Greedy-Ansatz nicht für eine optimale Lösung 
garantieren kann.

Die Zeitkomplexität des Greedy-Algorithmus zur Lösung des 
Rucksackproblems ist $O(n\log(n))$, 
wobei $n$ die Anzahl der Gegenstände darstellt. Dabei nimmt das 
Sortieren der Liste der Gegenstände die meiste Zeit ein, bzw.\ hat 
die Höchste Zeitkomplexität, weil das Sortieren einer Liste 
keine geringere Zeitkomplexität als $(n\log(n))$ haben kann\ \cite[vgl. S. 191--193][]{cormen2022introduction}.
Der Rest des Greedy"=Algorithmus hat eine 
Laufzeit von $O(n)$.\ \cite[vgl. S. 425ff.][]{cormen2022introduction}

\section{Dynamische Programmierung}
Um das Rucksackproblem mit dynamischer Programmierung zu 
lösen, müssen zuerst die Teilprobleme betrachtet werden. 
Sei $\dpd$ eine Tabelle, die die maximale 
Gesamtwertsumme für jede Teilmenge der Gegenstände und jede 
mögliche Kapazität des Rucksacks speichert. Die Einträge in 
der Tabelle werden durch $\dpd[i][j]$ repräsentiert, wobei $i$ den 
Index des aktuellen Gegenstands und $j$ die aktuelle Kapazität 
des Rucksacks darstellt.

Listing\ \ref{dp01} zeigt den Algorithmus beispielhaft in Python Code implementiert.
Dabei ist $\dpd$ die oben beschriebene Tabelle als zweidimensionale Liste 
implementiert. Der Basisfall tritt ein, falls $i$ oder $j$ gleich $0$ werden, 
das bedeutet, das entweder ein nicht existierender Gegenstand abgefragt wird 
($i = 0$) oder kein Platz mehr im Rucksack ist ($j = 0$).

Um den Wert $\dpd[i][j]$ zu berechnen vergleicht der Algorithmus den 
gesamten Profit, um zu entscheiden ob er den Gegenstand aufnimmt (erster Funktionsaufruf in \verb+max()+), 
oder nicht (zweiter Funktionsaufruf). Der Algorithmus nimmt den Gegenstand auch nicht auf, 
falls er nicht mehr in den Rucksack passt. Dabei überspringt der Algorithmus die 
Berechnung, falls schon ein Wert in $\dpd[i][j]$ eingetragen ist (Prinzip der 
dynamischen Programmierung). $w$ ist die Liste mit den Gewichten und $p$ die 
Liste mit den Profiten der Gegenstände, dessen Referenzen im code in\ \ref{dp01} 
immer weitergegeben werden.\\

\vbox{
\begin{lstlisting}[language=python, caption=Python code für Dynamische Programmierung, label=dp01][H!]
def dp_knapsack(dp, i, j, w, p):
    # Basisfall
    if i == 0 or j == 0:
        return 0
    if not dp[i][j]:
        if w[i] <= j:
            dp[i][j] = max(dp_knapsack(dp, i-1, j, w, p),
                           dp_knapsack(dp, i-1, j-w[i], w, p) + p[i])
        else:
            dp[i][j] = dp_knapsack(dp, i-1, j, w, p)
    return dp[i][j]
\end{lstlisting}
}

\noindent Der Wert in $\dpd[1][1]$ gibt die maximale Gesamtwertsumme an und 
repräsentiert somit die optimale Lösung des Rucksackproblems.

Die Laufzeit dieses Algorithmus beträgt $O(nc)$, weil 
eine Tabelle der Größe $(n+1) x (c+1)$ erstellt wird und 
jeder Eintrag der Tabelle in konstanter Zeit berechnet wird.
Weil der Algorithmus damit (auch) von der Größe einer Zahl abhängt, 
hat er eine \textit{pseudopolynomielle} Zeit-Komplexität.

Der größte Nachteil der Lösung mit dynamischer Programmierung ist, dass 
sie nur für ganzzahlige Gewichte funktioniert, weil es keine nicht-ganzzahligen 
indexe gibt.\ \cite[vgl.][]{Martello1987}